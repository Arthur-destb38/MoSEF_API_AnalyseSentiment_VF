% ============================================================
% PARTIE SCRAPING (Arthur) -- Simple : 4 slides, tableau recapitulatif
% ============================================================

\section{II. Scraping}

% ---------- Slide 1 : Objectif + pipeline en largeur + notions réduites ----------
\begin{frame}{Scraping -- Objectif}
\vspace{0.15cm}
\centering
\begin{tikzpicture}[
  node distance=0.55cm,
  box/.style={draw=#1, line width=1.2pt, rounded corners=10pt, minimum width=2.1cm, minimum height=0.95cm, fill=bglight, align=center, font=\small},
  arrow/.style={-{Stealth[length=2.5mm]}, thick, accent}
]
  \node[box=accent] (sources) {\faIcon{comments}\\[-2pt]\textbf{Sources}\\[-4pt]\scriptsize Reddit, Twitter, etc.};
  \node[box=accent, right=0.6cm of sources] (collecte) {\faIcon{cloud-download-alt}\\[-2pt]\textbf{Collecte}\\[-4pt]\scriptsize Scraping / API};
  \node[box=border, right=0.6cm of collecte] (stockage) {\faIcon{database}\\[-2pt]\textbf{Stockage}\\[-4pt]\scriptsize SQLite};
  \node[box=success, right=0.6cm of stockage] (nlp) {\faIcon{brain}\\[-2pt]\textbf{NLP}\\[-4pt]\scriptsize Sentiment};
  \node[box=success, right=0.6cm of nlp] (viz) {\faIcon{chart-line}\\[-2pt]\textbf{Dashboard}\\[-4pt]\scriptsize Streamlit};
  \draw[arrow] (sources) -- (collecte) (collecte) -- (stockage) (stockage) -- (nlp) (nlp) -- (viz);
\end{tikzpicture}
\vspace{0.5cm}
\begin{block}{Objectif}
\centering
Collecter des posts texte depuis plusieurs plateformes pour alimenter l'analyse de sentiment crypto.
\end{block}
\vspace{0.2cm}
\textcolor{textsecondary}{\small \textbf{Notions :} Sources multiples (Reddit, Twitter, YouTube\ldots) ; collecte HTTP / Selenium / APIs ; sauvegarde auto en base.}
\end{frame}

% ---------- Slide 2 : Tableau récapitulatif (remarques courtes, lignes alternées) ----------
\begin{frame}{Scraping -- Tableau récapitulatif}
\vspace{0.15cm}
\small
\begin{center}
\begin{tabular}{@{}llp{5.2cm}@{}}
\toprule
\textbf{Source} & \textbf{Méthode} & \textbf{Remarque} \\
\midrule
Reddit & HTTP ou Selenium & Fallback Selenium si bloqué \\
\rowcolor{bglight}
Twitter & Selenium & Profils publics ou login \\
StockTwits & Selenium & Labels Bullish/Bearish (NLP) \\
\rowcolor{bglight}
Telegram & API / HTTP & Channels publics, texte \\
4chan & HTTP & Forum /biz/, requêtes directes \\
\rowcolor{bglight}
Bitcointalk & HTTP & Forum crypto, scraping pages \\
GitHub & API & Issues, dépôts crypto \\
\rowcolor{bglight}
Bluesky & AT Protocol & Posts publics \\
YouTube & API & Commentaires (clé API) \\
\bottomrule
\end{tabular}
\end{center}
\vspace{0.3cm}
\textcolor{textsecondary}{\scriptsize Limites : Reddit HTTP $\sim$1000 posts, Selenium $\sim$200. StockTwits = seule source avec labels humains.}
\end{frame}

% ---------- Slide 3 : Schéma des méthodes (HTTP / Selenium / API) ----------
\begin{frame}{Scraping -- Méthodes : qui fait quoi ?}
\vspace{0.3cm}
\centering
\begin{tikzpicture}[
  scale=1.05,
  every node/.append style={transform shape},
  box/.style={draw=#1, line width=1.2pt, rounded corners=10pt, minimum width=2.2cm, minimum height=1cm, fill=bglight, align=center, font=\small},
  arrow/.style={-{Stealth[length=2.5mm]}, thick, accent},
  ex/.style={font=\scriptsize, text=textsecondary, align=left}
]
  \node[box=accent, minimum width=1.8cm] (plateforme) {\textbf{Plateforme}};
  \node[box=success, above right=1.2cm and 2.2cm of plateforme] (api) {\textbf{API}\\[-2pt]\scriptsize Rapide, clé requise};
  \node[box=accent, right=2.6cm of plateforme] (selenium) {\textbf{Selenium}\\[-2pt]\scriptsize Pas d'API / dynamique};
  \node[box=border, below right=1.2cm and 2.2cm of plateforme] (http) {\textbf{HTTP}\\[-2pt]\scriptsize Pages statiques};
  \draw[arrow] (plateforme) -- (api);
  \draw[arrow] (plateforme) -- (selenium);
  \draw[arrow] (plateforme) -- (http);
  \node[ex, right=0.45cm of api, anchor=west] {Reddit, Telegram, GitHub,\\YouTube, Bluesky};
  \node[ex, right=0.45cm of selenium, anchor=west] {Twitter, StockTwits};
  \node[ex, right=0.45cm of http, anchor=west] {4chan, Bitcointalk};
\end{tikzpicture}
\end{frame}

% ---------- Slide 4 : Anecdotes terrain (format court, tout visible) ----------
\begin{frame}{Scraping -- Anecdotes terrain}
\vspace{0.4cm}
\begin{itemize}
  \item \textbf{\faIcon{key} YouTube} — Clé API sur \textbf{Google Cloud Console} (YouTube Data API v3) ; quotas gratuits limités.
  \vspace{0.25cm}
  \item \textbf{\faIcon{shield-alt} Twitter} — Anti-bot \textbf{renforcé très récemment} $\to$ Selenium aléatoire ; en pratique : profils publics ou compte dédié.
  \vspace{0.25cm}
  \item \textbf{\faIcon{bug} Nitter} — Instances souvent down ou bloquées ; à ne pas utiliser comme seule source Twitter.
  \vspace{0.25cm}
  \item \textbf{\faIcon{chrome} Chrome} — \textbf{Fermer toutes les fenêtres} avant un scrape Selenium (sinon le driver se connecte à une session existante).
  \vspace{0.25cm}
  \item \textbf{\faIcon{chart-line} StockTwits} — Seule source avec \textbf{labels humains} Bullish/Bearish pour évaluer les modèles NLP.
\end{itemize}
\end{frame}

