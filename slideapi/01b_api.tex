% ============================================================
% PARTIE API -- Endpoints, GET, POST, etc.
% ============================================================

\section{III. API}

% ---------- Slide 1 : Notions GET, POST, endpoint (texte) ----------
\begin{frame}{API -- Notions : GET, POST, endpoint}
\vspace{0.25cm}
\textbf{GET} -- Lire des donn\'ees.
\begin{itemize}
  \item M\'ethode HTTP pour \textbf{r\'ecup\'erer} des informations (état du serveur, prix, liste\ldots).
  \item Pas de corps (body) : les param\`etres passent dans l'URL ou en \textbf{query} (\texttt{?source=reddit\&limit=50}).
  \item Exemples : \texttt{GET /health}, \texttt{GET /prices/bitcoin}, \texttt{GET /limits}.
\end{itemize}
\vspace{0.3cm}
\textbf{POST} -- Envoyer des donn\'ees.
\begin{itemize}
  \item M\'ethode HTTP pour \textbf{envoyer} des donn\'ees au serveur (lancer un scrape, analyser du texte\ldots).
  \item Les donn\'ees sont envoy\'ees dans le \textbf{body} de la requête, en général en \textbf{JSON} (ex.\ \texttt{\{"source": "reddit", "crypto": "bitcoin", "limit": 50\}}).
  \item Exemples : \texttt{POST /scrape}, \texttt{POST /sentiment}, \texttt{POST /analyze}.
\end{itemize}
\vspace{0.3cm}
\textbf{Endpoint} (ou \textbf{route}) = m\'ethode HTTP (GET ou POST) + \textbf{chemin} (\texttt{/health}, \texttt{/scrape}, \texttt{/prices/\{crypto\}}). C'est l'adresse de la ressource ou de l'action sur l'API.
\end{frame}

% ---------- Slide 2 : Notre API -- Vue d'ensemble des endpoints ----------
\begin{frame}{API -- Notre API (Crypto Sentiment)}
\vspace{0.15cm}
\textbf{FastAPI} : documentation auto (\texttt{/docs}), validation Pydantic, async.

\vspace{0.3cm}
\small
\begin{center}
\begin{tabular}{@{}llp{5.5cm}@{}}
\toprule
\textbf{Méthode} & \textbf{Endpoint} & \textbf{Rôle} \\
\midrule
GET & \texttt{/} & Page d'accueil (HTML) \\
GET & \texttt{/health} & Health check \\
GET & \texttt{/limits} & Limites scraping par plateforme \\
GET & \texttt{/prices/\{crypto\}} & Prix actuel (CoinGecko) \\
\rowcolor{bglight}
POST & \texttt{/scrape} & Scraper une source (body: source, crypto, limit) \\
POST & \texttt{/sentiment} & Analyser des textes (body: texts, model) \\
POST & \texttt{/analyze} & Pipeline scraping + sentiment \\
POST & \texttt{/compare/models} & Comparer FinBERT vs CryptoBERT \\
\rowcolor{bglight}
GET & \texttt{/storage/stats} & Stats base \\
GET & \texttt{/storage/export/csv} & Export CSV \\
GET & \texttt{/storage/export/json} & Export JSON \\
\bottomrule
\end{tabular}
\end{center}
\vspace{0.2cm}
\textcolor{textsecondary}{\scriptsize GET = lecture ; POST = action avec body JSON. Doc interactive : \texttt{/docs}.}
\end{frame}

% ---------- Slide 3 : Exemples concrets ----------
\begin{frame}{API -- Exemples concrets}
\vspace{0.15cm}
\begin{block}{GET (pas de body)}
\texttt{GET /health} $\to$ \texttt{\{"status": "ok", "timestamp": "\ldots"\}}\\
\texttt{GET /prices/bitcoin} $\to$ \texttt{\{"id": "bitcoin", "symbol": "btc", "current\_price": \ldots\}}\\
\texttt{GET /limits} $\to$ limites par source (reddit, twitter, youtube\ldots)
\end{block}
\vspace{0.2cm}
\begin{block}{POST (body JSON)}
\texttt{POST /scrape} avec \texttt{\{"source": "reddit", "crypto": "bitcoin", "limit": 50\}} $\to$ nb posts, sample, temps.\\
\texttt{POST /sentiment} avec \texttt{\{"texts": ["BTC to the moon"], "model": "finbert"\}} $\to$ \texttt{label}, \texttt{score} par texte.\\
\texttt{POST /analyze} : scraping + analyse sentiment en une requête.
\end{block}
\vspace{0.15cm}
\textcolor{textsecondary}{\scriptsize En pratique : \texttt{curl}, Postman, ou le dashboard Streamlit qui appelle ces endpoints.}
\end{frame}
